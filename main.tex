\documentclass{article}
\usepackage[utf8]{inputenc}
\usepackage{amsmath}
\usepackage{amsfonts}
\usepackage{appendix}

\title{The n-Volume Formula}
\author{Jack Hudson}
\date{September 2020}

% Convert to Two Separate Papers

\begin{document}

\maketitle

\[
\int_S \star \sqrt{f^{\star}g}
\]

\section{Introduction}


\newpage
\tableofcontents
\newpage

\section{Deriving the Generalized n-Volume Formula}
\subsection{The Generalized Volume Element}
For any given volume, its n-volume\footnote{The n-volume is the generalized volume for n-dimensions. A 1-volume is a length, a 2-volume a surface area, and a 3-volume a volume. In general, an n-volume describes the size of a surface.} is given by $\int_S dV$, where $S$ is the domain of surface, and $dV$ is the n-volume element of the space. For any coordinate system, the volume element is given by $\sqrt{det(\eta)}$, followed by a chain of coordinate differentials for each coordinate describing a unique point in the volume. Here, $\eta$ is the metric of the space, which is the Minkowski Metric, as we will assume space to be flat for now.

In order to multiply by all coordinate differentials, the language of differential forms can be taken advantage of. Because $\sqrt{det(\eta)}$ is a pure function, it is in effect a 0-form. Applying the Hodge Star Operator to it, then, will yield its D-form twin, where D is the number of dimensions of the surface. For a three-dimensional surface parameterized by the coordinates $(x,y,z)$, the generalized volume element is then given by
\[
\star \sqrt{det(\eta)} = \sqrt{det(\eta)} dx \wedge dy \wedge dz
\]
and so the volume occupied by this surface is given by
\[
\int\int\int_S \sqrt{det(\eta)} dx \wedge dy \wedge dz
\]
where S is the domain of all three coordinates.

For a surface area or length, the same general formula can be applied. However, two or one coordinates would coordinatize the space, respectively, meaning that the integral would be over two or one coordinates.

\subsection{Pullback of the Metric Tensor}
The metric $\eta$ describes flat coordinates limiting it to only finding the volumes of rectangular prisms, the areas of rectangles, and the lengths of straight lines. However, any given surface has its own coordinates so long as it is parameterized. Therefore, the metric of that surface can be found via a pullback of the metric of the space the surface is embedded in. The simplest procedure for this is to find the space's coordinates in terms of the surface's coordinates, then to find the new metric in terms of the surface's coordinates. If the surface can be described by the mapping
\[
f(u,v): (x,y,z) \rightarrow (x(u,v), y(u,v), z(u,v))
\]
then the pullback of the metric $\eta$ by the mapping $f$ can be denoted $f^{\star}g$. In this example, a two dimensional surface is embedded in a three dimensional space. However, the pullback method to find the surface's metric will work for an any dimensional surface embedded in an any dimensional space.

For example, a surface parameterized by the coordinates $u$ and $v$ embedded in two dimensional Euclidean space would have coordinate differentials
\[
dx = \frac{\partial x}{\partial u}du + \frac{\partial x}{\partial v} dv, dy = \frac{\partial y}{\partial u}du + \frac{\partial y}{\partial v} dv
\]
Therefore, the metric:
\[
ds^2 = dx \otimes dx + dy \otimes dy
\]
Becomes
\[
ds^2 = (\frac{\partial x}{\partial u}du + \frac{\partial x}{\partial v} dv) \otimes (\frac{\partial x}{\partial u}du + \frac{\partial x}{\partial v} dv) + (\frac{\partial y}{\partial u}du + \frac{\partial y}{\partial v} dv) \otimes (\frac{\partial y}{\partial u}du + \frac{\partial y}{\partial v} dv)
\]
This new metric can now be directly plugged into the generalized n-volume formula:
\[
V = \int_S \star \sqrt{det(f^{\star}g)}
\]
\subsection{Procedure}
The procedure for utilizing the n-volume formula is relatively straightforward. It is as follows:
\begin{itemize}
    \item Find the parameterizing map.
    \item Find the coordinate differentials in terms of the parametric coordinates.
    \item Find the metric tensor in terms of the new coordinate differentials.
    \item Find the square root of the determinant of the metric tensor.
    \item Plug this into the integral and evaluate.
\end{itemize}
The remainder of this paper will be the process of following this procedure for a number of situations, and interpriting the results.

\newpage

\section{Flat Space}
The n-volume formula can be applied to a surface of any dimension embedded in a space of any dimension. However, in many trivial circumstances, it generalizes to simpler, common formulas. In the following section, the n-volume formula is used to derive some common geometric formulas.

\subsection{Reduction to the Arc Length Formula}
We will now set about to derive the arc length formula from the n-volume formula. Let $f(x)$ be an arbitrary function. As a map, this can be given by:
\[
f(t):(x,y)\rightarrow(x(t),y(t))
\]
Where $t=x$ and $y(t)=f(x)$. The 2D Minkowski metric is, of course:
\[
dx \otimes dx + dy \otimes dy
\]
Applying the pullback, the coordinate differentials are as follows:
\[
dx = \frac{\partial x}{\partial t} dt = dt, dy = \frac{\partial y}{\partial t} dt = f'(t) dt
\]
Therefore, the pullback of the metric by the function is
\[
dx \otimes dx + dy \otimes dy = dt \otimes dt + (f'(t))^2 dt \otimes dt = (1 + (f'(t))^2) dt \otimes dt
\]
The determinant of this metric is (as it is, as a matrix, a 1x1 matrix) simply $(1 + (f'(t))^2)$. Taking the square root of this yields $\sqrt{(1 + (f'(t))^2)}$. This is the $\sqrt{det(f^{\star}g)}$ part of the n-volume formula. Plugging it in yields
\[
\int_S \star \sqrt{(1 + (f'(t))^2)}
\]
The curve only has a singular dimension (t), making the Hodge star operator simply yield
\[
\int_S \sqrt{(1 + (f'(t))^2)}dt
\]
Because we have imposed $x=t$, we can replace $t$ by $x$ get the formula
\[
\int_{x_i}^{x_f} \sqrt{(1 + (f'(x))^2)}dx
\]
Here we have also added in the domain $S$. This formula is exactly the arc length formula!

QED

\subsubsection{Parametric Curves}
In the previous part, there was no motivation to set $x=t$. Let us now not make that assumption. The arbitrary map is now $f(t):(x,y)\rightarrow(x(t),y(t))$. The pullback of the metric now yields the coordinate differentials:
\[
dx = \frac{\partial x}{\partial t} dt, dy = \frac{\partial y}{t} dt
\]
The metric is then:
\[
dx \otimes dx + dy \otimes dy = (\frac{\partial x}{\partial t})^2 dt \otimes dt + (\frac{\partial y}{\partial t})^2 dt \otimes dt = ((\frac{\partial x}{\partial t})^2 + (\frac{\partial y}{\partial t})^2) dt \otimes dt
\]
The determinant of this metric is of course $(\frac{\partial x}{\partial t})^2 + (\frac{\partial y}{\partial t})^2$. Therefore, plugging that into the n-volume formula gives
\[
\int_S \star \sqrt{(\frac{\partial x}{\partial t})^2 + (\frac{\partial y}{\partial t})^2} = \int_{t_i}^{t_f} \sqrt{(\frac{\partial x}{\partial t})^2 + (\frac{\partial y}{\partial t})^2} dt
\]
Again, the domain $S$ has been inserted as the bounds of the integral. This is exactly the parametric arc length formula!

\subsection{Surface Areas of Three Dimensional Shapes}
The n-volume formula can be used to find surface areas as easily as it can be used to find lengths. It will now be applied to a number of common shapes in three dimensions in order to derive the formulas for their surface areas.

\subsubsection{The 2-Sphere}
A 2-sphere embedded in three dimensional space can be described by the parameterizing map $f(\theta,\phi):(x,y,z)\rightarrow(Rcos(\theta)sin(\phi),Rsin(\theta)sin(\phi),Rcos(\phi))$ for $\theta\in[0,\pi]$, $\phi\in[0,2\pi)$. Pulling back the metric by this parameterization yields the coordinate differentials:
\[
dx = \frac{\partial x}{\partial \theta} d\theta + \frac{\partial x}{\partial \phi} d\phi = -Rsin(\theta)sin(\phi) d\theta + Rcos(\theta)cos(\phi) d\phi
\]
\[
dy = \frac{\partial y}{\partial \theta} d\theta + \frac{\partial y}{\partial \phi} d\phi = Rcos(\theta)sin(\phi) d\theta + Rsin(\theta)cos(\phi) d\phi
\]
\[
dz = \frac{\partial z}{\partial \theta} d\theta + \frac{\partial z}{\partial \phi} d\phi = -Rsin(\phi) d\phi
\]
The full metric with pullback is extremely unwieldy, so the terms will be calculated individually.
\[
dx \otimes dx = R(-sin(\theta)sin(\phi) d\theta + cos(\theta)cos(\phi) d\phi) \otimes R(-sin(\theta)sin(\phi) d\theta + cos(\theta)cos(\phi) d\phi)
\]
\[
= R^2sin^2(\theta)sin^2(\phi) d\theta \otimes d\theta + R^2cos^2(\theta)cos^2(\phi) d\phi \otimes d\phi - 2R^2sin(\theta)sin(\phi)cos(\theta)cos(\phi) d\theta \otimes d\phi
\]
\[
dy \otimes dy = R(cos(\theta)sin(\phi) d\theta + sin(\theta)cos(\phi) d\phi) \otimes R(cos(\theta)sin(\phi) d\theta + sin(\theta)cos(\phi) d\phi)
\]
\[
= R^2cos^2(\theta)sin^2(\phi) d\theta \otimes d\theta + R^2sin^2(\theta)cos^2(\phi) d\phi \otimes d\phi + 2R^2cos(\theta)sin(\phi)sin(\theta)cos(\phi) d\theta \otimes d\phi
\]
\[
dz \otimes dz = R^2sin^2(\phi) d\phi \otimes d\phi
\]
Summing them all together and enacting some cancellations:
\[
R^2(sin^2(\theta)sin^2(\phi) + cos^2(\theta)sin^2(\phi)) d\theta \otimes d\theta + R^2(cos^2(\theta)cos^2(\phi) + sin^2(\theta)cos^2(\phi) + sin^2(\phi)) d\phi \otimes d\phi
\]
\[
= R^2sin^2(\phi) d\theta \otimes d\theta + R^2d\phi \otimes d\phi
\]
The determinant of this matrix is simply the product of the coefficients, as it is a diagonal metric:
\[
det(f^{\star}\eta) = R^4sin^2(\theta)
\]
Therefore the n-volume formula is
\[
\int_S \star R^2sin(\theta) = \int_S R^2sin(\theta) d\theta \wedge d\phi
\]
$S$ is the domain of the angles of the surface of a sphere, which we have already asserted to be $\theta\in[0,\pi]$, $\phi\in[0,2\pi)$. Therefore, the full integral is:
\[
\int_{0}^{2\pi}\int_{0}^{\pi}R^2sin(\theta) d\theta \wedge d\phi
\]
This integral will be computed explicitly, though going forward integral solutions will simply be given.
\[
2R^2\int_{0}^{2\pi}d\phi = 4\pi R^2
\]
Which is exactly the standard formula from geometry. QED

\subsubsection{The Torus}
The torus is defined by the following parameterization:
\[
f(\theta,\phi): (x,y,z) \rightarrow ((a cos(\theta)+b)cos(\phi), (a cos(\theta)+b)sin(\phi), a sin(\theta))
\]
\[
\theta, \phi \in [0, 2\pi)
\]
Here, $b$ is the larger radius of the torus, and $a$ is the smaller radius. The coordinate differentials are then:
\[
dx = (b cos\phi - a cos\phi sin\theta)d\theta - (a cos\theta+b)sin\phi d\phi
\]
\[
dy = (b cos\phi - a sin\phi sin\theta)d\theta + (a cos\theta + b)cos\phi d\phi
\]
\[
dz = a cos\theta d\theta
\]
Meaning:
\[
f^{\star}\eta = a^2 d\theta \otimes d\theta + (a cos \theta + b)^2 d\phi \otimes d\phi
\]
Giving an overall n-volume integral:
\[
\int_{0}^{2\pi}\int_{0}^{2\pi}(a^2 cos\theta + ab) d\theta \wedge d\phi = 4\pi^2ab
\]
This is, then, the formula for the surface area of the torus.

\subsubsection{The Cylinder}
The cylinder (more specifically, its side) is defined by the parameterizing map:
\[
f(\theta, z): (x,y,z) \rightarrow (R cos\theta, R sin\theta, z)
\]
The coordinate differentials are relatively straightforward:
\[
dx = -R sin(\theta) d\theta, dy = R cos(\theta) d\theta, dz = dz
\]
Therefore, the metric is:
\[
R^2 sin^2(\theta) d\theta \otimes d\theta + R^2 cos^2(\theta) d\theta \otimes d\theta + dz \otimes dz = R^2 d\theta \otimes d\theta + dz \otimes dz
\]
The n-volume formula is then simply
\[
\int_S \star R = \int_{0}^{h}\int_{0}^{2\pi} R d\theta \wedge dz = 2\pi Rh
\]
Here, $h$ is the height of the cylinder we are considering. This is, of course, the expected formula for its surface area.

\subsubsection{The Cone}
A cone with height $h$ and circular radius $R$ is parameterized by the map:
\[
R(\theta,t): (x,y,z) \rightarrow (Rt cos\theta, Rt sin\theta, h(1-t))
\]
\[
\theta \in [0, 2\pi), t \in [1, 0]
\]
Here, t is essentially the scale factor of the radius as one moves up the cone. As always, one must proceed by working out the coordinate differentials:
\[
dx = -Rt sin\theta d\theta + R cos\theta dt
\]
\[
dy = Rt cos\theta d\theta + R cos\theta dt
\]
\[
dz = -hdt
\]
\[
f^{\star}\eta = R^2 t^2 d\theta \otimes d\theta + (R^2 + h^2) dt \otimes dt
\]
The square root of the determinant of this metric is then:
\[
\sqrt{R^2t^2(R^2+h^2)} = Rt\sqrt{R^2+h^2}
\]
This $\sqrt{R^2+h^2}$ has a straightforward geometric interpritation: the slant height! It may be easier, then, to replace that radical with the slant height $l$, which will be used along with $R$ to differentiate between different cones. The n-volume formula is then:
\[
\int_S \star Rtl = \int_1^0\int_0^{2\pi} Rtl d\theta \wedge dt = -\int_0^1\int_0^{2\pi} Rtl d\theta \wedge dt
\]
This will, though, yield a negative area. Clearly, this is undesirable. The issue results from how the manifold (the cone) is oriented with respect to our coordinates. Reversing the orientation will cause the expression to pick up another negative sign, which will cancel the one existing, giving us the desired positive area. This can be done by simply reversing the order to differentiation, as the exterior product $\wedge$ is anti-symmetric. Therefore, the area is given by:
\[
\int_0^1\int_0^{2\pi} Rtl dt \wedge d\theta = \pi Rl
\]
As always, exactly as it should be.

\subsection{Volumes Contained in Three Dimensional Shapes}
\subsubsection{The Sphere}
The Minkowski Metric in spherical coordinates is given by:
\[
\eta = dr \otimes dr + r^2 d\theta \otimes d\theta + r^2 sin^2\theta d\phi \otimes d\phi
\]
In these coordinates, the parameterizing map for the interior of a sphere is:
\[
f(r, \theta, \phi):(r,\theta, \phi) \rightarrow (r, \theta, \phi)
\]
The pullback of the metric, then, is itself! This greatly simplifies calculations. The determinant of this metric is:
\[
r^2 sin^2 (\theta)
\]
Meaning that the n-volume formula yields:
\[
\int_S \star R^2 sin(\theta) = \int_{0}^{2\pi}\int_{0}^{\pi}\int_{0}^{R}R^2 sin(\theta) dr \wedge d\theta \wedge d\phi = \frac{4}{3}\pi R^3
\]
This is exactly the expected volume for a sphere.

\subsubsection{The Torus}
The interior of the torus is defined by:
\[
f(r,\theta,\phi): (x,y,z) \rightarrow ((r cos(\theta)+b)cos(\phi), (r cos(\theta)+b)sin(\phi), r sin(\theta))
\]
The coordinate differentials are:
\[
dx = cos(\theta)cos(\phi)dr - r sin(\theta) cos(\phi) d\theta - (r cos(\theta) + b) sin(\phi) d\phi
\]
\[
dy = cos(\theta)sin(\phi) dr - r sin(\theta) sin(\phi) d\theta + (r cos(\theta) + b) cos(\phi) d\phi
\]
\[
dz = sin(\theta) dr + r cos(\theta) d\theta
\]
Unfortunately, three dimensions can become rather complicated when dealing with cross-terms. However, the metric tensor is a rank-2 tensor, meaning that it can be represented as a matrix (specifically a 3x3 matrix). This will make it much simpler to see terms and how they cancel when we start adding together the coordinate differentials.
\[
dx \otimes dx =
\begin{bmatrix}
cos^2\theta cos^2\phi & -r sin\theta cos \theta cos^2 \phi & -(r cos\theta + b) cos\theta sin\phi cos\phi \\
-r cos\theta cos\theta cos^2\phi & r^2 sin^2\theta cos^2 \phi & r(r cos\theta + b)sin\theta sin\phi cos\phi \\
-(r cos\theta + b) cos\theta sin\phi cos\phi & r(r cos\theta + b)sin\theta sin\phi cos\phi & (r cos\theta + b) sin^2\phi
\end{bmatrix}
\]
\[
dy \otimes dy =
\begin{bmatrix}
cos^2\theta sin^2\phi & -r sin\theta cos \theta sin^2 \phi & (r cos\theta + b) cos\theta sin\phi cos\phi \\
-r sin\theta cos\theta sin^2\phi & r^2 sin^2\theta sin^2 \phi & -r(r cos\theta + b)sin\theta sin\phi cos\phi \\
(r cos\theta + b) cos\theta sin\phi cos\phi & -r(r cos\theta + b)sin\theta sin\phi cos\phi & (r cos\theta + b) cos^2\phi
\end{bmatrix}
\]
\[
dz \otimes dz =
\begin{bmatrix}
sin^2\theta & r sin\theta cos\theta & 0 \\
r sin\theta cos\theta & r^2 cos^2\theta & 0 \\
0 & 0 & 0
\end{bmatrix}
\]
These matrices are, evidently, extremely ugly. However, summing them leads to all of the off-diagonals cancelling, and many terms simplifying from simple trigonometric identities. The pulled-back metric is:
\[
f^{\star}\eta = \begin{bmatrix}
1 & 0 & 0 \\
0 & r^2 & 0 \\
0 & 0 & (r cos\theta + b)^2
\end{bmatrix}
\]
Therefore, the n-volume formula yields:
\[
\int_S \star (r^2 cos\theta + rb) = \int_0^{2\pi}\int_0^{2\pi}\int_0^{R}(r^2 cos\theta + rb) dr \wedge d\theta \wedge d\phi
\]
Splitting up this integral, it is easy to see that the first term will be 0. (This is because the integral of cosine from 0 to 2$\pi$ is 0.) Therefore, the integral simplifies to
\[
\int_0^{2\pi}\int_0^{2\pi}\int_0^{R} rb dr \wedge d\theta \wedge d\phi = 2\pi^2R^2b
\]
Which is exactly the volume within a torus.

\subsubsection{The Cylinder}
Using the Jacobian method\footnote{See [Appendix A]}, the cylinder's volume is exceedingly easy to compute. The parameterizing map is:
\[
f(r, \theta, z): (x, y, z) \rightarrow (r cos \theta, r sin \theta, z)
\]
\[
r \in [0, R], \theta \in [0, 2\pi), z \in [0, h]
\]
The Jacobian matrix is, then:
\[
J_{ij} = \begin{bmatrix}
cos \theta & -r sin \theta & 0 \\
sin \theta & r cos \theta & 0 \\
0 & 0 & 1
\end{bmatrix}
\]
Making the Jacobian $J = r cos^2 \theta + r sin^2 \theta = r$. The n-volume formula is then:
\[
\int_S \star J = \int_0^h\int_0^{2\pi}\int_0^R r dr \wedge d\theta \wedge dz = \pi R^2 h
\]
Exactly as expected.

\subsubsection{The Cone}
The final "normal" computation (normal here meaning in 3 dimensions and fewer, in flat space) we will undergo is the volume of the cone. Again, the Jacobian method will be used for its computational simplicity.

Our parameterizing map will be:
\[
f(r, \theta, z): (x, y, z) \rightarrow ((1-\frac{z}{h})Rr cos\theta, (1-\frac{z}{h})Rr sin\theta, z)
\]
\[
r \in [0, 1], \theta \in [0, 2\pi), z \in [0, h]
\]
Here, we have chosen a slightly different convention for $r$. The issue is that the distance one can move from the $r=0$ line depends on where they are in the cone. At the cone's point ($z=h$), $r$ is stuck at 0. At the base ($z=0$), $r$ can range from 0 to $R$. Therefore, we shall take the convention that $r$ now ranges from 0 to 1, and it represents the percentage of the possible distance from the $r=0$ line. Therefore, the surface of the cone is defined by $r=1$. As it turns out, this makes little difference in the calculation, it simply changes the integration bounds.

The Jacobian matrix is given by
\[
J_{ij} = \begin{bmatrix}
(1-\frac{z}{h})Rcos \theta & -(1-\frac{z}{h})Rr sin \theta & 0 \\
(1-\frac{z}{h})Rsin \theta & (1-\frac{z}{h})Rr cos \theta & 0 \\
0 & 0 & 1
\end{bmatrix}
\]
Giving a Jacobian of $J=(1-\frac{z}{h})Rr$. Therefore, the n-volume formula yields:
\[
\int_S \star J = \int_0^h\int_0^{2\pi}\int_0^1 (1-\frac{z}{h})Rr dr \wedge d\theta \wedge dz
\]
\[
= \int_0^h\int_0^{2\pi}\int_0^1 Rr dr \wedge d\theta \wedge dz - \int_0^h\int_0^{2\pi}\int_0^1 \frac{z}{h} Rr dr \wedge d\theta \wedge dz
\]
\[
= \pi R^2 h - \frac{1}{2} \pi R^2 h = \frac{1}{2} \pi R^2 h
\]
The volume of the cone is, then, exactly half the volume of the cylinder that it would fit inside.

\subsubsection{The One-Sheeted Hyperboloid}


\newpage

\section{Curved Space}
Now that the n-Volume formula has been proven to work in the simplest of cases, it can be applied to more exotic scenarios: curved spaces.

\subsection{Pullback of a Curved Metric}
The means by which a curved metric is pulled back is essentially identical to the flat (Euclidean) metric. First, the metric is explicitly identified:
\[
g = X(x,y,z) dx \otimes dx + Y(x,y,z) dy \otimes dy + Z(x,y,z) dz \otimes dz
\]
Then, the parameterizing map $f$ is also explicitly defined, as well as all of the parametrized coordinates. The coordinate differentials $dx$, $dy$, and $dz$ are then identified in terms of the parameterized coordiantes, and the metric is then found in terms of those coordinates. Finally, the metric is inserted into the n-volume formula, and all that remains to be done is the evaluation of the resulting integral.

\newpage
\subsection{Path Length}
\subsubsection{Standard Metric}
Many space-time metrics from General Relativity (specifically spherically symmetric ones) take the form
\[
g = -f(r) dt \otimes dt + f(r)^{-1} dr \otimes dr + r^2 d\Omega_2 \otimes d\Omega_2
\]
Because we are only considering the sizes of shapes, time is unchanging, reducing the standard metric to
\[
g = f(r)^{-1} dr \otimes dr + r^2 d\Omega_2 + d\Omega_2
\]
Because many metrics take this form, it is a much more reasonable endeavor to work out the lengths, areas, and volumes of shapes in terms of any general $f(r)$, and then to simply insert the proper $f(r)$ for any particular space and evaluate, once the specific space is known. Below is a table of $f(r)$ expressions and the spaces they represent:
\[
\begin{tabular}{|c|c|}
\hline
    Expression & Space \\ \hline
    $1$ & Minkowski/Euclidean Space \\ \hline
    $1 - \frac{2GM}{r}$ & Schwartzchild Space \\ \hline
    $1 - \frac{r^2}{\alpha^2}$ & de Sitter Space \\ \hline
    $1 + \frac{r^2}{\alpha^2}$ & Anti-de Sitter Space \\ \hline
\end{tabular}
\]

Now, to find the path length. A general path in polar coordinates can be represented by the map:
\[
\mu(t): (r, \theta, \phi) \rightarrow (R(t), \Theta(t), \Phi(t))
\]
Here, $\mu$ represents the map, as $f$ is already taken, as part of the metric. Speaking of, the metric is the standard metric
\[
g = \frac{1}{f(r)} dr \otimes dr + r^2 d\Omega_2 \otimes \Omega_2
\]
The coordinate differentials are the straightforward:
\[
dx = (R'(t))^2 dt, dy = (\Theta'(t))^2 dt, dz = (\Phi'(t))^2 dt
\]
Therefore, the metric is:
\[
g = (\frac{1}{f(r)}(R'(t))^2 + r^2 (\Theta'(t))^2 + r^2 sin^2(\Theta(t))(\Phi'(t))^2) dt \otimes dt
\]
This has only one element as a matrix, yielding a straightforward determinant. The n-volume formula is therefore:
\[
\int_S \sqrt{(\frac{1}{f(r)}(R'(t))^2 + r^2 (\Theta'(t))^2 + r^2 sin^2(\Theta(t))(\Phi'(t))^2)} dt
\]
However, it is rare that a path is specified with angles as functions of time, except for very specific circumstances. Therefore, it is useful to convert now to Cartesian coordinates, via the definitions:
\[
r = \sqrt{x^2 + y^2 + z^2}, \theta = arctan(\frac{y}{x}), \phi = arctan(\frac{\sqrt{x^2 + y^2}}{z})
\]
Plugging these definitions into the formula whole, however, would result in a monstrously long expression, and so it is simply necessary to convert on a case-by-case basis. The generalized length formula we have now cannot be analyzed any further without specifying a particular spacial geometry.

\subsubsection{Schwartzschild Space}
\subsubsection{de Sitter Space}
\subsubsection{anti-de Sitter Space}

\newpage
\subsection{The Sphere and the Shell}
\subsubsection{Standard Metric}
\subsubsection{Schwartzschild Space}
\subsubsection{de Sitter Space}
\subsubsection{anti-de Sitter Space}

\newpage
\subsection{The Torus}
\subsubsection{Standard Metric}
\subsubsection{Schwartzschild Space}
\subsubsection{de Sitter Space}
\subsubsection{anti-de Sitter Space}

\newpage
\subsection{The Cylinder}
\subsubsection{Standard Metric}
\subsubsection{Schwartzschild Space}
\subsubsection{de Sitter Space}
\subsubsection{anti-de Sitter Space}

\newpage
\subsection{The Cone}
\subsubsection{Standard Metric}
\subsubsection{Schwartzschild Space}
\subsubsection{de Sitter Space}
\subsubsection{anti-de Sitter Space}

\newpage
\subsection{The One-Sheeted Hyperboloid}
\subsubsection{Standard Metric}
\subsubsection{Schwartzschild Space}
\subsubsection{de Sitter Space}
\subsubsection{anti-de Sitter Space}

\newpage
\subsection{Kerr Space}
\subsubsection{Path Length}
\subsubsection{The Sphere}
\subsubsection{The Torus}
\subsubsection{The Cylinder}
\subsubsection{The Cone}
\subsubsection{The One-Sheeted Hyperboloid}

\newpage
\appendix
\section{The Jacobian Alternative Formulation}


\section{Deriving the Metric of the Hypersphere}


\end{document}
